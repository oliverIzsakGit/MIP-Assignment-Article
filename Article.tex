
% Metódy inžinierskej práce

\documentclass[10pt,twoside,english,a4paper]{article}

\usepackage[english]{babel}
%\usepackage[T1]{fontenc}
\usepackage[IL2]{fontenc} % lepšia sadzba písmena Ľ než v T1
\usepackage[utf8]{inputenc}
\usepackage{graphicx}
\usepackage{url} % príkaz \url na formátovanie URL
\usepackage{hyperref} % odkazy v texte budú aktívne (pri niektorých triedach dokumentov spôsobuje posun textu)

\usepackage{cite}
%\usepackage{times}


\pagestyle{plain}

\title{The use of Synchronous and asynchronous e-learning methods efficiency comparison\thanks{Semestrálny projekt v predmete Metódy inžinierskej práce, ak. rok 2015/16, vedenie: 	
Ing. Jozef Sitarčík}} % meno a priezvisko vyučujúceho na cvičeniach

\author{Olivér Izsák\\[2pt]
	{\small Slovenská technická univerzita v Bratislave}\\
	{\small Fakulta informatiky a informačných technológií}\\
	{\small \texttt{xizsak@stuba.sk}}
	}

\date{\small 15. October 2020} 



\begin{document}

\maketitle

\begin{abstract}
\ldots
People have been trying to prove one method's superiority over the other in order to use the better method for teaching.
Although in reality neither one is better than the other. Both methods should be used in e-learning , however
in different ways  and in different scenarios. The main purpose of this article is to discuss  where and when to use asynchronous and
synchronous e-learning methods to achieve efficient e-learning. It is based mainly on two studies, Stefan Hrastinski's study of "Asynchronous and Synchronous E-Learning"~\ref{source1}, and the other is Ayesha Perveen's study: "Synchronous and Asynchronous E-Language Learning:A Case Study of Virtual University of Pakistan"~\ref{source2}
The goal of this article is to show the pros and cons of the two learning methods, and when and how to use them.
\end{abstract}



\section{Introduction} \label{intro}

Online learning environments have become widespread phenomenon in 21st century, but their efficiency has been a controversial topic between scientist for decades. Especially in the early days of Online e-learning, where online telecommunication application such as Skype,Discord,Google Meet etc... were not widespread nor available for an average person for either technical or financial reasons,  that meant real-time face-to-face communication was not possible, which made e-learning less desirable, since Synchronous learning was not available yet. However in the modern days, we have a great amount of online telecommunication applications and improvements in technology and increasing bandwidth capabilities have led to the growing popularity of e-learning, therefore both Asynchronous and Synchronous e-learning is possible nowadays.
Synchronous learning happens in real-time with immediate interaction between the participants, which has been considered by many as the superior e-learning method, for a while because of the social aspect of learning, even though studies suggest that is not entirely accurate.
On the other hand, Asynchronous e-learning is self-paced and not time bound, which means that the individual can learn whenever he wants or feels like it.This has been around for time in a form of Distance education or distance learning.The article is based on one small scale scale study by Stefan Hrastinski and one large scale case study done by Ayesha Perveen , who based study on the Virtual University of Pakistan. The article will also explain the Synchronous e-learning method~\ref{sync}.
Asynchronous e-learning method~\ref{async}, Benefits and Disadvantages of both methods,~\ref{badose}~\ref{badoae}
and the most efficient way to use them together to achieve high level e-learning~\ref{comb}. 
Finally, in the conclusion ads dw will be discussed~\ref{conc}

\footnote{Distance education : also called distance learning is the education of students who may not always be physically present at school, through mail or other methods.}





\section{Synchronous E-learning} \label{sync}

Z obr.~\ref{f:rozhod} je všetko jasné. 

\begin{figure*}[tbh]
\begin{center}
    \includegraphics[scale=1.0]{diagram.pdf}
%Aj text môže byť prezentovaný ako obrázok. Stane sa z neho označný plávajúci objekt. Po vytvorení diagramu zrušte znak \texttt{\%} pred príkazom \verb|\includegraphics| označte tento riadok ako komentár (tiež pomocou znaku \texttt{\%}).
\caption{Rozhodujúci argument.}
\label{f:rozhod}
\end{center}


List of applications for synchronous e-learning:
\begin{itemize}
\item jedna vec
\item druhá vec
	\begin{itemize}
	\item x
	\item y
	\end{itemize}
\end{itemize}

\end{figure*}



\section{Asynchronous E-learning} \label{async}

Základným problémom je teda\ldots{} Najprv sa pozrieme na nejaké vysvetlenie (časť~\ref{ina:nejake}), a potom na ešte nejaké (časť~\ref{ina:nejake}).

Môže sa zdať, že problém vlastne nejestvuje\cite{Coplien:MPD}, ale bolo dokázané, že to tak nie je~\cite{Czarnecki:Staged, Czarnecki:Progress}. Napriek tomu, aj dnes na webe narazíme na všelijaké pochybné názory\cite{PLP-Framework}. Dôležité veci možno \emph{zdôrazniť kurzívou}.


\subsection{Nejaké vysvetlenie} \label{ina:nejake}



Ten istý zoznam, len číslovaný:

\begin{enumerate}
\item jedna vec
\item druhá vec
	\begin{enumerate}
	\item x
	\item y
	\end{enumerate}
\end{enumerate}


\subsection{Types of commu} \label{ina:este}

\paragraph{Veľmi dôležitá poznámka.}
Niekedy je potrebné nadpisom označiť odsek. Text pokračuje hneď za nadpisom.



\section{Benefits and disadvantages of synchronous E-learning} \label{badose}




\section{Benefits and disadvantages of Asynchronous E-learning} \label{badoae}

\section{Efficiency of the Combination of Asynchronous and Synchronous E-learning } \label{comb}


\section{Conclusion} \label{conc} % prípadne iný variant názvu



%\acknowledgement{Ak niekomu chcete poďakovať\ldots}


% týmto sa generuje zoznam literatúry z obsahu súboru literatura.bib podľa toho, na čo sa v článku odkazujete
\bibliography{literatura}
\bibliographystyle{alpha} % prípadne alpha, abbrv alebo hociktorý iný
\end{document}
